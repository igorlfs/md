\documentclass{article}

\usepackage{indentfirst} %% Indente o primeiro parágrafo
\usepackage{amsfonts} %% Conjuntos
%%\usepackage{etoolbox} ?
\usepackage{amssymb}
\usepackage{amssymb} %% QED
\usepackage{enumitem}
\usepackage{graphicx} %% Imagens
\usepackage{float} %% Coloque imagens em lugares apropriados, ie H
\usepackage[T1]{fontenc}        % Encoding para português 
\usepackage{lmodern}            % Conserta a fonte para PT
\usepackage[portuguese]{babel}  % Português
\usepackage{hyphenat}           % Use hífens corretamente

\usepackage[margin=1.5in]{geometry}


\graphicspath{ {./img} }
\hyphenation{mate-mática recu-perar}
\setlist{  listparindent=\parindent }

\title{Lista 6: Notação Grande O}

\author{Igor Lacerda}

\begin{document}

\maketitle

\section*{Nota}

Nessa lista, eu (tentei) usar \LaTeX\ em todo o documento, mas foi uma experiência nada divertida que se provou extremamente trabalhosa. Nas listas seguintes vou voltar a fazer a parte discursiva a punho mesmo.

\section*{Questões Discursivas (QD)}

\begin{enumerate}

	\item A definição de \( f(n) \in O(g(n)) \) é:

	      \[ f(n) \in O(g(n)) \iff \exists c, n_0 \mid \forall n > n_0 : f(n) \leq c \cdot g(n) \]

	\item Se \( f_1(n) = O(g_1(n)) \land f_2(n) = O(g_2(n))  \), temos que:

	      \[ f_1(n) + f_2(n) = O(\textrm{max}(g_1(n), g_2(n)) )  \]

	\item A ordem da soma de dois polinômios é a ordem do polinômio de maior grau.

	\item Se \( f_1(n) = O(g_1(n)) \land f_2(n) = O(g_2(n))  \), temos que:

	      \[ f_1(n) \cdot f_2(n) = O(g_1(n) \cdot g_2(n)) \]

	\item Uma função \( f(n) \) pertence a \( \Omega(g(n)) \) se:

	      \[ f(n) = \Omega(g(n)) \iff g(n) = O(f(n)) \]

	\item Uma função \( f(n) \) pertence a \( \Theta(g(n)) \) se:

	      \[ f(n) = \Theta(g(n)) \iff g(n) = O(f(n)) \land f(n) = O(g(n))\]

	\item É comum ver a notação \( O(n) \) sendo usada no lugar da notação \( \Theta(n) \), quando se está falando da complexidade de um algoritmo. Essa distinção é importante. Eu poderia dizer que a vasta maioria dos algoritmos conhecidos é \( O(n!!) \), mas isso não é verdade para usando a notação teta. Nesse sentido, a notação com \( \Theta(n) \) é muito mais informativa. Além disso, é importante usar \( \in \) ao invés de \( = \).

\end{enumerate}

\section*{Exercícios}

\begin{enumerate}

	\item \( \models 3x^4 + 1 \in O(x^4/2) \land x^4/2 \in O(3x^4+1)\)

	      Pela QD 5, \( O(3x^4 + 1 ) = O(3x^4) = O(x^4) = O(x^4/2)\). Também usando uns princípios básicos, temos \( O(x^4/2) = O(x^4) = O(3x^4) = O(3x^4 +1) \).

	\item \( \models x \log x \in O(x^2) \land x^2 \not \in O(x \log x) \)

	      Suponha \( x \geq 1 \). Então temos: \( \log x \leq x \Rightarrow x \log x \leq x^2\), então para \( n_0 = 1 \land k = 1 \), temos que \( x \log (x) \in O(x^2) \). Suponha por contradição que \( x^2 \in O(x\log (x)) \). Então, para \( x \neq 0 \), \( \exists n_0, c \mid \forall x > n_0, x^2 \leq c \cdot x\log (x) \Rightarrow x \leq c \cdot \log (x) \Rightarrow x/\log (x) \leq c\), o que é impossível pois \( \not \exists c \mid c \geq x/\log (x) \forall x \).

	\item

	      \begin{enumerate}

		      \item \( (n^2 + 8)(n+1) \)

		            \( O(n^3) \), pois é um polinômio de grau 3.

		      \item \( (n \log n + n^2)(n^3 + 2) \)

		            \( O(n^5) \), pois o primeiro fator é \( O(n^2) \), pelo exercício 2, o que nos leva a um produto de polinômios, que segue a regra do caso anterior.

		      \item \( (n! + 2^n ) (n^3 + \log( n^2 + 1 )) \)

		            O primeiro fator é \( O(n!) \), pela regra da soma. O Segundo fator é \( O(n^3) \). Como \( n^2 + 1 \approx n^2 \) para \( n \) grande, podemos supor \( n^2 \), o que nos leva a \( \log n^2 \), ou \( 2\log n \), cuja complexidade polinomial é \( O(n) \). Então o produto é \( O(n^3 \cdot n!) \).

	      \end{enumerate}

	\item

	      \begin{enumerate}

		      \item \( n \cdot \log(n^2 + 1) + n^2 \cdot \log n\)

		            Assumindo uma definição muito específica para ``função simples'', aplicamos o raciocínio do item anterior: \( O(n^3) \).

		      \item \( (n \cdot \log n + 1)^2 + (\log n + 1)(n^2 + 1)\)

		            \(  (\log n + 1)(n^2 + 1)\) é \( O(\log n \cdot n^2) \). O outro lado é \( O(n^2 \cdot \log^2 n) \). Assim, o primeiro lado domina assintoticamente o segundo. O fato é que \( \log^2 n \) é \( O(n) \), então temos que a função é \( O(n^3) \).

		      \item \( n^{2^n} + n^{n^2} \)

		            Essa função é \( O(n^{2^n}) \), pela regra da soma.

	      \end{enumerate}

	\item \textit{Sejam \( f,g,h : \mathbb{N} \to \mathbb{R}_{ \geq 0} \). Prove a transitivada da notação O(): Se \( f(n) \in O(g(n)) \land g(n) \in O(h(n)) \) então \( f(n) \in O(h(n)) \)}

	      Temos \( \exists a_0, k \mid \forall n > a_0, f(n) \leq k \cdot g(n)\), e \( \exists b_0, l \mid \forall n > b_0, g(n) \leq l \cdot h(n) \). Então para \( c_0 = \textrm{max}(a_0,b_0) \), temos que \( f(n) \leq k \cdot g(n) \leq k \cdot l \cdot h(n) \), logo \( f(n) \) pertence a \( O(h(n)) \).

	\item Isso é permitido pois ao se trabalhar com a notação big-O, multiplicar por constantes não altera a ordem da função. Então, em vez de trabalhar com todos os inteiros, podemos supor que eles são pares.

	\item Ora, suponha que \( 2^n \) e \( 3^n \) são de mesma ordem. Em particular, que \( \exists n_0, k \mid \forall n > n_0, 3^n \leq k \cdot 2^n \) . Então \( (\frac{3}{2})^n \leq k \), o que é absurdo pois estaríamos tomando um \( k \) arbitrariamente grande.

	\item \[ O(n \cdot \log n) \subsetneq O(n^{1 + \epsilon }) \subsetneq O(n^2 / \log n) \subsetneq O((n^2 - n - 100)^4) = O(n^8) \subsetneq O((1+\epsilon)^n) \]

	      Por ordem, temos:

	      \[ \lim_{n \to \infty} \frac{n \log n}{n^{1 + \epsilon}} = \lim_{n \to \infty} \frac{\log n}{n^{\epsilon}} =\lim_{n \to \infty} \frac{1}{n \cdot \epsilon \cdot n^{\epsilon - 1}} =  0 \]

	      Usando L'Hôpital na última igualdade. Similarmente:

	      \[ \lim_{n \to \infty} \frac{n^{1 + \epsilon}}{\frac{n^2}{\log n} } = \lim_{n \to \infty} \frac{\log n \cdot n^{\epsilon}}{n} = \lim_{n \to \infty} \frac{\log n}{n^{1 - \epsilon}} = \lim_{n \to \infty} \frac{1}{(1 - \epsilon ) \cdot n^{-\epsilon} \cdot n} = 0 \]

	      Felizmente os outros casos não tão trabalhosos. De fato, é óbvio que não tem como \( O(n^2/ \log n) \) nem chegar perto de \( O(n^8) \), pelos graus dos polinômios envolvidos. Mesmo apelando a \( O(1/(\log n)) \in O(n)\), chegaríamos num \( O(n^3) \), que ainda está distante de \( O(n^8) \). Por outro lado, temos dois polinômios de grau 8, então ambos são de ordem \( n^8 \). Para completar, no final temos um exponencial com base \( > 1 \), que sabemos que cresce mais que qualquer polinômio.

	\item \textbf{NOOP}

	\item \textbf{NOOP}

\end{enumerate}

\end{document}
