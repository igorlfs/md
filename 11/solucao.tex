\documentclass{article}

%%\usepackage{indentfirst} %% Indente o primeiro parágrafo
\usepackage{amsfonts} %% Conjuntos
%%\usepackage{etoolbox} ?
\usepackage{amsmath}
\usepackage{amsthm}
\usepackage{amssymb} %% QED
\usepackage{enumitem}
\usepackage{graphicx} %% Imagens
\usepackage{float} %% Coloque imagens em lugares apropriados, ie H
\usepackage[T1]{fontenc}        % Encoding para português 
\usepackage{lmodern}            % Conserta a fonte para PT
\usepackage[portuguese]{babel}  % Português
\usepackage{hyphenat}           % Use hífens corretamente

\usepackage[margin=1.5in]{geometry}

\newcommand{\Mod}[1]{\ (\mathrm{mod}\ #1)} %% Facilita o módulo

\makeatletter
\let\saveqed\qed
\renewcommand\qed{%
   \ifmmode\displaymath@qed
   \else\saveqed
   \fi}


\graphicspath{ {./img} }
\hyphenation{mate-mática recu-perar}
\setlist[enumerate]{wide=\parindent}

\title{Lista 11: Grafos II}

\author{Igor Lacerda}

\begin{document}

\maketitle

\section*{Questões Discursivas}

\begin{enumerate}
	\item Grafos.
	      \begin{enumerate}

		      \item Um \textbf{ciclo Hamiltoniano} em um grafo simples é um ciclo que passa por todos os vértices uma única vez.

		      \item Um grafo de com um vértice de grau um não pode ter um ciclo hamiltoniano.

	      \end{enumerate}

	\item Grafos.

	      \begin{enumerate}

		      \item Um grafo é planar se pode ser desenhado no plano de tal modo que não haja interseções entre as arestas (exceto os próprios vértices).

		      \item \( K_5 \).

	      \end{enumerate}

	\item Grafos.
	      \begin{enumerate}

		      \item \( r = e - v + 2 \), em que \( r \) é o número de regiões, \( e \) o número de arestas e \( v \) o número de vértices.

		      \item A partir da fórmula de Euler pode se deduzir outras fórmulas, como \( e \leq 3v - 6 \), que todo grafo planar deve seguir. Então se um grafo não atende à fórmula, ele é não planar.

	      \end{enumerate}

	\item \textbf{Teorema de Kuratowski}: um grafo não é planar \textbf{sse} ele contiver um subgrafo homeomorfo a \( K_{3,3} \) ou \( K_5 \). Assim, somente os grafos que podem ser reduzidos a um destes grafos são não planares.

	\item Grafos.
	      \begin{enumerate}

		      \item \textbf{O número cromático} de um grafo é o menor número de cores necessárias para a coloração deste grafo. O número cromático de um grafo G é indicado por \( \chi (G) \).

		      \item \( \chi (K_n) = n \)

		      \item \( n \equiv 0 \textrm{ mod } 2 \Rightarrow \chi (C_n) = 2 \land n \equiv 1 \textrm{ mod } 2 \Rightarrow \chi (C_n) = 3\)

		      \item \( \chi (K_{m,n}) = 2 \)

	      \end{enumerate}

	\item \textbf{Teorema das quatro cores:} o número cromático de um grafo planar não é maior do que quatro. Sim, existem grafos cujo número cromático é maior que 4, no entanto eles não são planares.

	\item A coloração de grafos pode ser usada para modelar o mínimo de horários de aplicações de exames de um grupo de estudantes, ou pode ser usada em compiladores para fazer o registro de índices (veja o livro para uma discussão mais detalhada).


\end{enumerate}

\section*{Exercícios}

\end{document}
