\documentclass{article}

\usepackage{indentfirst} %% Indente o primeiro parágrafo
\usepackage{amsfonts} %% Conjuntos
%%\usepackage{etoolbox} ?
\usepackage{amssymb}
\usepackage{amssymb} %% QED
\usepackage{enumitem}
\usepackage{graphicx} %% Imagens
\usepackage{float} %% Coloque imagens em lugares apropriados, ie H
\usepackage[T1]{fontenc}        % Encoding para português 
\usepackage{lmodern}            % Conserta a fonte para PT
\usepackage[portuguese]{babel}  % Português
\usepackage{hyphenat}           % Use hífens corretamente

\graphicspath{ {./img} }
\hyphenation{mate-mática recu-perar}
\setlist{  listparindent=\parindent }

\title{Lista 5: Probabilidade III}

\author{Igor Lacerda}

\begin{document}

\maketitle

\section*{Questões discursivas}

\begin{enumerate}

    \item Considere \( X \) como uma variável aleatória no espaço amostral \( S \). Definimos a \textbf{variância} de \( X \), indicada por \( V(x) \), como

        \begin{center}
            \fbox{\begin{minipage}{15em}
                    \[ V(x) = \sum_{s \in S} (X(S) - E(X))^{2} \cdot p(s) \] 
            \end{minipage}}
        \end{center}

        O \textbf{desvio padrão} de \( X \), indicado por \( \sigma(x) \), é definido como \( \sqrt{ V(x) } \).

        Note que calcular a \textbf{variância} pela definição é um processo meio chatinho. Então podemos usar o seguinte teorema:

        \begin{center}
            \fbox{\begin{minipage}{11em}
                    \( V(x) = E(X^{2}) - E(X)^{2} \) 
            \end{minipage}}
        \end{center}

        De forma menos abstrata, a variância indica o ``o quão bem'' as probabilidades estão distribuídas. Quanto menor (maior) é a variância, mais próximas (distantes) estão as probabilidades.

    \item As variáveis aleatórias \( X, Y \) do espaço amostral \( S \) são \textbf{independentes} se 

        \[ P(X = x \land Y = y) = P(X = x) \cdot P(Y = y) \] 

        \footnote{Usando a notação do prof, que é menos carregada, mas com o discurso na do livro.} Ou, discursivamente, se a probabilidade de \( X(s) = r_1 \) e \( Y(s) = r_2  \) for igual ao produto das probabilidades de \( X(s) = r_1 \) e \( Y(s) = r_2 \), para todos os números reais \( r_1 \textrm{ e } r_2 \).

        O legal de se ter variáveis aleatórias \( X \), \( Y \) de um mesmo espaço amostral, é o seguinte teorema: \( E(XY) = E(X) \cdot E(Y) \).

    \item O \textbf{valor esperado} de uma distribuição normal corresponde ao ponto que é o \textit{pico do gráfico}, que é chamado de \( \mu \). Seu \textbf{desvio-padrão} é \( \sigma \), que representa os pontos de inflexão (quanto menor, mais próximos os valores estão).

    \item Na distribuição normal, o ``volume'' de probabilidade que corresponde ao intervalo \( [\mu - \sigma, \mu + \sigma] \) é \( \approx 68\% \), e no intervalo \( [\mu - 2\sigma, \mu + 2\sigma] \) é \( \approx 95\% \).

    \item A distribuição normal é uma aproximação da distribuição binomial, para \( n \) suficientemente grande.

    \item Na nova abordagem do \textbf{Teorema de Bayes}, nós usamos a \textit{chance} ao invés da probabilidade. A \textbf{chance} difere da probabilidade no seguinte sentido: enquanto a probabilidade é um número que varia entre 0 e 1 (incluindo os extremos), a chance é uma proporção do \textit{total de possibilidades que um evento ocorre e as que não ocorre.} Nisso, simplificamos um pouco as contas do Teorema de Bayes, pois não precisamos calcular \( p(E) \) (calculamos, por exemplo \( \frac{p(F \mid E)}{p(\overline{F} \mid E)} \)), que pode ser trabalhoso às vezes.

    \item No \textbf{Silogismo do Policial}, um policial \textit{infere} que uma certa pessoa cometeu um crime, dado que ela é suspeita e que o policial possui certa \textit{experiência} para averiguar essa questão. Não é uma \textit{dedução lógica} ``clássica'', de fato, esse raciocínio segue de uma extensão da lógica tradicional que usa probabilidade para concluir novas informações. 

    \item Essa pergunta apareceu numa lista anterior. Veja a questão 10, da lista 3.

\end{enumerate}

\section*{Exercícios}

Dessa vez, alguns estão anexados como fotos, mas outros estão feitos aqui mesmo, usando o \LaTeX.

\begin{enumerate}

    \item  Dê um exemplo de um algoritmo probabilístico não mencionado na aula.

        \textbf{R}: O \textbf{Algoritmo de Karger}, que faz \textit{alguma coisa} com grafos têm uma versão probabilística.

\end{enumerate}

\begin{center}
    \textit{Refazendo exercícios 6.3.4 e 6.3.10 usando o novo Bayes} 
\end{center}

\textbf{6.3.4.} C := Caixa, L := Laranja, P := Preta;

\[ C_1 = [3L, 4P], C_2 = [5L, 6P] \]

Quero saber a \textbf{probabilidade de \( C_2 \) dado L}.

\begin{center}
    \begin{tabular}{c | c  c | c} 
        & \(E = L\) & \( \overline{E} = P\) & \\ [0.5ex]
        \hline
        \( F = C_1 \) & 3 & 4 & \( Pr(F) = \frac{1}{2} \)  \\ [0.8ex]
        \hline
        \( \overline{F} = C_2 \) & 5 & 6 & \( Pr(\overline{F}) = \frac{1}{2} \) \\ [0.8ex]
    \end{tabular}
\end{center}

Ou seja, quero saber \( p(\overline{F} \mid E) \). Por Bayes:

\[ \frac{p(\overline{F} \mid E)}{p(F \mid E)} = \frac{p (E \mid \overline{F}) \cdot p(\overline{F} )}{p (E \mid {F}) \cdot p({F})} \] 


\[ \Rightarrow \frac{p(\overline{F} \mid E)}{p(F \mid E)} = \frac{5/11 \cdot 1/2}{3/7 \cdot 1/2} \] 

\[ \Rightarrow \frac{p(\overline{F} \mid E)}{p(F \mid E)} = \frac{5/11}{3/7} \]

\[ \Rightarrow {p(\overline{F} \mid E)} = \frac{5/11}{3/7 + 5/11} \]

\[ \Rightarrow {p(\overline{F} \mid E)} = \frac{5/11}{68/77} \]

\[ \Rightarrow {p(\overline{F} \mid E)} \approx 0.5147 \]

Uau, eu inclusive \textit{errei} a última passagem quando fiz essa problema pela primeira vez.

\textbf{6.3.10.} População: 4\% gripada; Teste: 97\% positivos verdadeiros e 2\% falsos positivos.

\begin{center}
    \begin{tabular}{c | c  c | c} 
        & \( F \) = Gripado & \( \overline{F} \) = Saudável & Total \\ [0.5ex]
        \hline
        \( E = + \) & 388 & 192 & 580 \\ 
        \hline
        \( \overline{E} = - \) & 12 & 9408 & 9420 \\ 
        \hline
        Total & 400 & 9600 & 10000 \\ [0.8ex]
    \end{tabular}
\end{center}

\begin{enumerate}

    \item \( p(F \mid E) \)

        \[ \frac{p(\overline{F} \mid E)}{p(F \mid E)} = \frac{192/9600 \cdot 96\%}{388/400 \cdot 4\%} \] 

        \[ \Rightarrow \frac{p(\overline{F} \mid E)}{p(F \mid E)} = \frac{192/9600 \cdot 24}{388/400} \] 

        \[ \Rightarrow {p(F \mid E)} = \frac{388/400}{192/9600 \cdot 24 + 388/400} \] 

        \[ \Rightarrow {p(F \mid E)} \approx 0.66\] 

    \item \( p(\overline{F} \mid E) \)

        Pelo exercício anterior:

        \[ \Rightarrow {p(F \mid E)} = \frac{ 24 \cdot 192/9600}{192/9600 \cdot 24 + 388/400} \] 

        \[ \Rightarrow {p(F \mid E)} \approx 0.33\] 

        Poderia também argumentar que é o complementar.

    \item \( p (F \mid \overline{E}) \)

        \[ \frac{p(F \mid \overline{E})}{p(\overline{F}  \mid \overline{E} )} = \frac{3\% \cdot 4\%}{9408/9600 \cdot 96\%} \] 

        \[ \Rightarrow \frac{p(F \mid \overline{E} )}{p(\overline{F} \mid \overline{E})} = \frac{3\%}{24 \cdot 9408/9600} \] 

        \[ \Rightarrow {p(F \mid \overline{E} )} = \frac{3\%}{24 \cdot 9408/9600 + 3\%} \] 

        \[ \Rightarrow {p(F \mid \overline{E} )} \approx 0.001\] 

    \item \( p (\overline{F}  \mid \overline{E}) \)

        Pelo complementar do anterior: 1 - 0.001 \( \approx \) 0.999.

\end{enumerate}

\textbf{Correção} 

\textbf{D.} Como João escolhe 13 cartas das \textit{51 restantes}, e há uma garantia que todas elas são do mesmo naipe (que não é o desejado), então Maria só pode ter escolhido uma entre opções de naipes, que são igualmente prováveis. Logo, a probabilidade é \( \frac{1}{3} \).

\textbf{E.} A probabilidade de o naipe das 13 cartas que João separou é \( \frac{1}{4} \), e nesse caso, é impossível Maria ter escolhido a carta do naipe desejado. Por outro lado, nos \( \frac{3}{4} \) restantes (João separou outro naipe), existe \( \frac{1}{3} \) de chance de a carta de Maria ser do naipe desejado. Assim, temos:

\[ \frac{1}{4} \cdot 0 + \frac{3}{4} \cdot \frac{1}{3} = \frac{1}{4} \] 

Assim, em princípio, é a mesma probabilidade que no caso base. O fato de João ter separado 13 cartas \textit{quaisquer} (no sentido que elas não agregam informação ao caso), não influencia no resultado final. Em contraste, com a \textbf{D}, em que é dada uma informação mais ``concreta''.

\end{document}
